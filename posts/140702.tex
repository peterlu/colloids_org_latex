\section{ACE-M2 first-round flight operations: major milestones achieved}\hypertarget{ace-m2-first-round-flight-operations-major-milestones-achieved}{}\label{ace-m2-first-round-flight-operations-major-milestones-achieved}
With ACE-M2, we sent up almost two dozen new, different samples, and did not
know \emph{a priori} if they would mix properly on orbit, how they would look in
the microscope, or whether we could observe their dynamics with the
instrumentation and constraints we have. After the first round of on-orbit
flight operations, we have a number of significant milestones---that we can mix,
load and image the samples; and that we now have flight operations procedures
that will give us good data going forward, within the envelope of our resources.
Over two 48-hour continuous runs (run 1 was 4-6 June 2014; run 2 was 18-20 June
2014), we were able to examine all of the samples (except for the well that
broke). Because we looked at different racks over each of these runs, I am
combining my discussion of the results treating this as one ``round'' of the
experiment. I will describe a few of the results from the samples that we have
thus far learned, in addition to everything else. As with most interesting
science experiments, the answers beget new questions, which is why this is an
exciting project to work on!

\subsection{Samples can be mixed and
imaged}\hypertarget{samples-can-be-mixed-and-imaged}{}\label{samples-can-be-mixed-and-imaged} As we now know from the first images, the mixing and optics are working well, so we can take
images and see all of the samples well. This is a major success and demonstrates
overall the samples are good, and that there is no obvious show-stopper
preventing us from gathering good data from our samples.

\emph{For reasons that I will elaborate upon in a later post, we are still coming up with a system to label and organize all of the image data, so this summary of results will wait until that process is finished before I add the processed images.}

\subsection{Dye samples}\hypertarget{dye-samples}{}\label{dye-samples}
We can image the fluorescence from all of our dye samples, using the Texas Red
and FITC filters, as appropriate. We don't see any structures in these samples,
so everything is as we expect. The field of illumination is not even, with sever
vignetting near the edges for the lowest magnifications. These samples which we
know to be spatially uniform might therefore provide a way to correct for
inhomogeneous illumination.

\subsection{Colloidal
suspensions}\hypertarget{colloidal-suspensions}{}\label{colloidal-suspensions} We can observe different levels of brightness for different volume fractions in
the simple particle suspensions. Ultimately, we want to be able to calibrate
overall brightness with volume fraction, so these samples will hopefully provide
a calibration mapping between image intensity and colloid volume fraction. This
is critically important to the science, and a major advance of the ACE
experiment over, say, previous BCAT iterations. We have always been interested
in measuring the volume fraction after phase separation is complete, but cannot
do so on the ground (because the phase separation process is different in
microgravity, which is why we do the experiment in the first place!). Being able
to measure that with ACE in our phase-separating samples would be a major
advance.

\subsubsection{Science is
serendipitous}\hypertarget{science-is-serendipitous}{}\label{science-is-serendipitous} Even when a sample may ``fail'' by our pre-existing criteria, we can still learn
interesting things, and turn that into a success by thinking along different
lines. One of the colloidal suspension samples appears to have had a stir bar
stuck in a dense suspension, very likely because the stir bar happened to be
stuck in a place around which colloid sedimented densely, preventing the bar
from being freed later on orbit. However, we know both the starting volume
fraction of the sample throughout the whole sample well, and the fraction of
that well occupied by dense colloid that sedimented. This allows us to estimate
the volume fraction, which we expect to be near to the hard-sphere glass
transition volume fraction of 58\%. That would give us an additional volume
fraction point on our calibration curve---one which is not possible to create
otherwise, because you physically could not load a colloidal suspension at that
density! Sometimes, even an initial failure can give you data you could not
otherwise acquire!

\subsection{Colloid-polymer
mixtures}\hypertarget{colloid-polymer-mixtures}{}\label{colloid-polymer-mixtures} We have three colloid-polymer mixture sample classes. One is a phase-separating
mixture of small particles and polymer; another is a phase-separating one with
large particles; and the third is an arrested gel (at least on earth) that
involves large particles and polymer.

\subsubsection{Phase separating
samples}\hypertarget{phase-separating-samples}{}\label{phase-separating-samples} Initial observations indicate that the mixing is sufficient, and we can watch
the evolution of structures in these samples over several days. The timescale is
quite appropriate for the ACE experiment: samples change over hours or days,
making useful the collection of similar sets of data over the course of days to
weeks. They are not changing so fast that we will miss the activity during
sample mixing and loading into the LMM; nor are they so slow that no activity is
discernable during our runs. This is all very good news.

\subsubsection{Gel sample}\hypertarget{gel-sample}{}\label{gel-sample}
Moveover, after waiting several weeks, we see the formation of stable colloidal
structure which may or may not be kinetically arrested; we are eagerly awaiting
new data on this particular point.

